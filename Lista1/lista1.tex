\documentclass[10pt,a4paper]{article}
%\usepackage[brazil]{babel}         
%\usepackage[latin1]{inputenc}      
%\usepackage[T1]{fontenc}            
\usepackage{graphicx}
\usepackage{epsfig}                     
\usepackage{epic,eepic}             
\usepackage{psfrag}                     
\usepackage[tight]{subfigure}   
\usepackage{tabularx}            
\usepackage{booktabs}        
\usepackage{multirow}          
\usepackage{longtable}         
\usepackage{array}   
\usepackage{amsmath}
\usepackage{xcolor}
\usepackage{fancyhdr}
\usepackage[portuges,brazilian]{babel}
%\usepackage[T1]{fontenc}
\usepackage{setspace}
\usepackage{indentfirst}
\usepackage{graphicx}
\usepackage{pslatex}
\usepackage[a4paper,  margin=0.5in]{geometry}


\begin{document}

   %\title{Modelos de Regress\~ao Linear - \\
   %        Lista de Exerc\'icios 1}   
   %\date{Abril 2015}
   %\maketitle 

   %\newpage

  \begin{center}
     \bf{UNIVERSIDADE FEDERAL DE GOI\'AS  \\ INSTITUTO DE MATEM\'ATICA E ESTAT\'ISTICA \\ LISTA 1}
  \end{center} 

  

  \begin{flushleft}
     {\bf Data: 16/03/2018 $\phantom{11111}$  Disciplina: T\'opicos em R $\phantom{11111}$  Prof. Dr. Renato Rodrigues Silva  }
  \end{flushleft}

  
 \small{  
  \begin{center}
     
     
   \begin{enumerate}
      \item Considere a seguinte matriz

           \[ \mathbf{X} = \left[\begin{array}{ccc}
                6 & 1 & 0 \\
                1 & 2 & 3 \\
                0 & 3 & 10 \end{array} \right] \]  
    
    Acesse os elementos da matriz $\mathbf{X}$ de tal forma que a submatriz resultante seja

           \[  \left[\begin{array}{ccc}
                6 & 0 \\
                0 & 10 \\
                \end{array} \right] \]   
      
     \item Determine a soma dos primeiros cinquenta elementos de uma sequência tal que 
     $a_n = 10n + 1, n \in N : n > 0$.
          
   
        \item Uma progress\~ao geom\'etrica p.g pode ser definida por meio de: $a_n = a_1  q^{n-1}$ em $n$ \'e o n\'umero de termos,
     $q$ \'e a raz\~ao $a_1$ \'e o primeiro termo da sequ\^encia e $a_n$ \'e o n-\'esimo termo da sequ\^encia.
     Fa\c{c}a uma p.g com $n=5$, $a_1 = 1$ e $q=2$


      \item Calcule o logaritmo natural de cada elemento da sequencia anterior 
       
      \item Defina a seguinte matriz
  
           \[ \mathbf{W} = \left[\begin{array}{cccccc}
                6 & 1 & 0  & 0 & 0 & ~0  \\
                1 & 2 & 3  & 0 & 0 & ~0  \\
                0 & 3 & 10 & 0 & 0 & ~0  \\   
                0 & 0 & 0  & 6 & 1 & ~0  \\
                0 & 0 & 0  & 1 & 2 & ~3  \\
                0 & 0 & 0  & 0 & 3 & 10  \\

\end{array} \right] \]  
      
   \end{enumerate}  
   Obs: Todos os exerc\'icios devem ser feitos utilizando o software R
 \end{center}

          




  

\end{document}


